
\documentclass[notheorems,serif]{beamer}

%选用主题
%\usetheme{Rochester}
%\usetheme{default}
%\usetheme{AnnArbor}
%\usetheme{Antibes}
%\usetheme{Bergen}
%\usetheme{Berkeley}
%\usetheme{Berlin}
%\usetheme{Boadilla}
%\usetheme{CambridgeUS}
%\usetheme{Copenhagen}
%\usetheme{Darmstadt}
%\usetheme{Dresden}
%\usetheme{Frankfurt}
%\usetheme{Goettingen}%
%\usetheme{Hannover}
%\usetheme{Ilmenau}
%\usetheme{JuanLesPins}
%\usetheme{Luebeck}
\usetheme{Madrid}
%\usetheme{Malmoe}
%\usetheme{Marburg}
%\usetheme{Montpellier}
%\usetheme{PaloAlto}
%\usetheme{Pittsburgh}
%\usetheme{Rochester}
%\usetheme{Singapore}
%\usetheme{Szeged}
%\usetheme{Warsaw}

% As well as themes, the Beamer class has a number of color themes
% for any slide theme. Uncomment each of these in turn to see how it
% changes the colors of your current slide theme.

%\usecolortheme{albatross}
%\usecolortheme{beaver}
%\usecolortheme{beetle}
%\usecolortheme{crane}
%\usecolortheme{dolphin}
%\usecolortheme{dove}
%\usecolortheme{fly}
%\usecolortheme{lily}
%\usecolortheme{orchid}
%\usecolortheme{rose}
%\usecolortheme{seagull}
%\usecolortheme{seahorse}
%\usecolortheme{whale}
%\usecolortheme{wolverine}

%设置被cover的内容不显示
%\setbeamercovered{transparent}

\useinnertheme{rounded}
\usecolortheme{default}

%调用包
\usepackage[no-math, cm-default]{fontspec}
\usepackage{xltxtra}
\usepackage{xunicode}   
\usepackage{xcolor}
\usepackage{amsmath,amssymb}
\usepackage{xeCJK}
\usepackage{multimedia}
\usepackage{listings}
\usepackage{subfigure}
\usepackage{todonotes}
\presetkeys{todonotes}{inline}{} 
\usepackage{multicol}
\usepackage{changes}




%将系统字体名映射为逻辑字体名称,主要是为了维护的方便  
\newcommand\fnhei{Adobe 黑体 Std}  
\newcommand\fnsong{Adobe 宋体 Std}  
\newcommand\fnkai{Adobe 楷体 Std}  
\newcommand\fnmono{DejaVu Sans Mono}  
\newcommand\fnroman{Times New Roman}  

\renewcommand{\normalsize}{\wuhao}

%%设置常用中文字号,方便调用  
\newcommand{\erhao}{\fontsize{22pt}{\baselineskip}\selectfont}  
\newcommand{\xiaoerhao}{\fontsize{18pt}{\baselineskip}\selectfont}  
\newcommand{\sanhao}{\fontsize{16pt}{\baselineskip}\selectfont}  
\newcommand{\xiaosanhao}{\fontsize{15pt}{\baselineskip}\selectfont}  
\newcommand{\sihao}{\fontsize{14pt}{\baselineskip}\selectfont}  
\newcommand{\xiaosihao}{\fontsize{12pt}{\baselineskip}\selectfont}  
\newcommand{\wuhao}{\fontsize{10.5pt}{\baselineskip}\selectfont}  
\newcommand{\xiaowuhao}{\fontsize{9pt}{\baselineskip}\selectfont}  
\newcommand{\liuhao}{\fontsize{7.5pt}{\baselineskip}\selectfont}  

%\setmainfont{\fnroman}
\setmainfont{\fnkai}
\setCJKmainfont[BoldFont=\fnhei]{\fnkai}  
\setCJKsansfont[BoldFont=\fnhei]{\fnkai}  
\setCJKmonofont{\fnkai}  

%楷体  
%\newfontinstance\KAI{\fnkai}  
%\newcommand{\kai}[1]{{\KAI#1}}  
%黑体  
%\newfontinstance\HEI{\fnhei}  
%\newcommand{\hei}[1]{{\HEI#1}}  
%英文  
%\newfontinstance\ENF{\fnroman}  
%\newcommand{\en}[1]{\,{\ENF#1}\,}

%楷体  
\newfontfamily\KAI {\fnkai}  
\newcommand{\kai}[1]{{\KAI#1}}  
%黑体  
\newfontfamily\HEI{\fnhei}  
\newcommand{\hei}[1]{{\HEI#1}}  
%英文  
\newfontfamily\ENF{\fnroman}  
\newcommand{\en}[1]{\,{\ENF#1}\,}


%连字符
\defaultfontfeatures{Mapping=tex-text}

%中文断行
\XeTeXlinebreaklocale "zh"
\XeTeXlinebreakskip = 0pt plus 1pt minus 0.1pt


%%%% 定理类环境的定义 %%%%
\newtheorem{example}{\hei{例子}} 
\newtheorem{problem}{\hei{问题}}           
\newtheorem{algorithm}{\hei{算法}}
\newtheorem{theorem}{\hei{定理}}
\newtheorem{definition}{\hei{定义}}
\newtheorem{axiom}{\hei{公理}}
\newtheorem{property}{\hei{性质}}
\newtheorem{proposition}{\hei{命题}}
\newtheorem{lemma}{\hei{引理}}
\newtheorem{corollary}{\hei{推论}}
\newtheorem{remark}{\hei{注解}}
\newtheorem{condition}{\hei{条件}}
\newtheorem{conclusion}{\hei{结论}}
\newtheorem{assumption}{\hei{假设}}

%重定义一些环境的名字
\renewcommand{\proofname}{\hei{证明}}
\renewcommand\tablename{\hei{表}}

%---SCRIPT-----------------------------------------------------------------------------------------
\newcommand{\cA}{\mathcal{A}}
\newcommand{\cB}{\mathcal{B}}
\newcommand{\cC}{\mathcal{C}}
\newcommand{\cD}{\mathcal{D}}
\newcommand{\cE}{\mathcal{E}}
\newcommand{\ce}{\mathcal{e}}
\newcommand{\cF}{\mathcal{F}}
\newcommand{\cG}{\mathcal{G}}
\newcommand{\cg}{\mathcal{g}}
\newcommand{\cH}{\mathcal{H}}
\newcommand{\cI}{\mathcal{I}}
\newcommand{\cJ}{\mathcal{J}}
\newcommand{\cK}{\mathcal{K}}
\newcommand{\cL}{\mathcal{L}}
\newcommand{\cM}{\mathcal{M}}
\newcommand{\cN}{\mathcal{N}}
\newcommand{\cO}{\mathcal{O}}
\newcommand{\cP}{\mathcal{P}}
\newcommand{\cQ}{\mathcal{Q}}
\newcommand{\cR}{\mathcal{R}}
\newcommand{\cS}{\mathcal{S}}
\newcommand{\cT}{\mathcal{T}}
\newcommand{\cU}{\mathcal{U}}
\newcommand{\cV}{\mathcal{V}}
\newcommand{\cW}{\mathcal{W}}
\newcommand{\cX}{\mathcal{X}}
\newcommand{\cY}{\mathcal{Y}}
\newcommand{\cZ}{\mathcal{Z}}
\newcommand{\cz}{\mathcal{z}}
%---BLACKBOARD-------------------------------------------------------------------------------------
\newcommand{\mA}{\mathbb A}
\newcommand{\mB}{\mathbb B}
\newcommand{\mC}{\mathbb C}
\newcommand{\mD}{\mathbb D}
\newcommand{\mE}{\mathbb E}
\newcommand{\mF}{\mathbb F}
\newcommand{\mG}{\mathbb G}
\newcommand{\mg}{\mathbb g}
\newcommand{\mH}{\mathbb H}
\newcommand{\mI}{\mathbb I}
\newcommand{\mJ}{\mathbb J}
\newcommand{\mK}{\mathbb K}
\newcommand{\mL}{\mathbb L}
\newcommand{\mM}{\mathbb M}
\newcommand{\mN}{\mathbb N}
\newcommand{\mO}{\mathbb O}
\newcommand{\mP}{\mathbb P}
\newcommand{\mQ}{\mathbb Q}
\newcommand{\mR}{\mathbb R}
\newcommand{\mS}{\mathbb S}
\newcommand{\mT}{\mathbb T}
\newcommand{\mU}{\mathbb U}
\newcommand{\mV}{\mathbb V}
\newcommand{\mW}{\mathbb W}
\newcommand{\mX}{\mathbb X}
\newcommand{\mY}{\mathbb Y}
\newcommand{\mZ}{\mathbb Z}
\newcommand{\mz}{\mathbb z}

\newcommand{\bV}{\mathbf{V}}
\newcommand{\bz}{\mathbf{z}}
\newcommand{\bT}{\mathbf{T}}
\newcommand{\bx}{\mathbf{x}}
\newcommand{\be}{\mathbf{e}}
\newcommand{\bff}{\mathbf{f}}
\newcommand{\bg}{\mathbf{g}}
\newcommand{\bn}{\mathbf{n}}
\newcommand{\bt}{\mathbf{t}}
\newcommand{\bd}{\mathbf{d}}
\newcommand{\bzero}{\mathbf{0}}
\newcommand{\bka}{\mathbf{\kappa}}

\newcommand{\rd}{\mathrm{d}}
%---SHORTCUTS--------------------------------------------------------------------------------------
\newcommand\xor{\mathbin{\char`\^}}
\DeclareMathOperator{\res}{Res}
\DeclareMathOperator{\sgn}{sgn}
\DeclareMathOperator{\supp}{supp}
\DeclareMathOperator{\as}{as}
\newcommand{\slant}[1]{\slshape #1\normalfont}
\newcommand{\dd}[2]{\frac{d#1}{d#2}} 
\newcommand{\ddx}{\frac{d}{dx}}
\newcommand{\ddt}{\frac{d}{dt}}
\newcommand{\dds}{\frac{d}{ds}}
\newcommand{\pd}[1]{\ds\frac{\partial}{\partial #1 }}
\newcommand{\pdd}[2]{\ds\frac{\partial #1}{\partial #2 }}
\newcommand{\mdd}[3]{\ds\frac{\partial^{#3} #1}{\partial #2^{#3} }}
\newcommand{\x}{\ _\Box}
\newcommand{\ds}{\displaystyle}
\newcommand{\bs}{\backslash}
\newcommand{\Bold}{\noindent \bfseries}
\newcommand{\Norm}{\normalfont}
\newcommand{\exl}[1]{\textcolor{NavyBlue}{\Bold Exercise #1 \Norm}}
\newcommand{\ex}{\textcolor{NavyBlue}{\Bold Problem: \Norm}}
\newcommand{\sol}{\textcolor{Mulberry}{\Bold Solution: \Norm}}
\newcommand{\pf}{\textcolor{Mulberry}{\Bold Proof: \Norm}}
\newcommand{\Title}[1]{\LARGE\Bold \textcolor{Sepia}{#1}\Norm\normalsize \vspace{10pt} \newline}
\newcommand{\prop}{\Bold \textcolor{YellowOrange}{ Proposition:} \Norm}
\newcommand{\propl}[1]{\Bold \textcolor{YellowOrange}{ Proposition #1:} \Norm}
\newcommand{\rk}{\Bold \textcolor{YellowOrange}{ Remark:} \Norm}
\newcommand{\rmk}[1]{\Bold\textcolor{YellowOrange}{#1} \Norm}
\newcommand{\thm}[1]{\Bold \textcolor{YellowOrange}{ Theorem #1} \Norm}
\newcommand{\ind}{\indent\indent}
\newcommand{\br}{\vspace{10pt} \newline}

\newcommand{\red}{\color{red}}
\newcommand{\blue}{\color{blue}}



\begin{document}

\title[数值线性代数]{{\small\leftline{数值线性代数———}~~~~~~~~~~~~~~~~~~~~~~~~~~~~~~~~~~~~~~~~~~~~~~
~~~~~~~~~~~} \\
矩阵计算
}




\author[]{~~潘建瑜~~}

\institute[湘潭大学数学系]

\date[\today]

%\pgfdeclareimage[height=1cm]{institution-logo}{figures/xtu.pdf}

%\logo{\pgfuseimage{institution-logo}}


\frame[plain]{\titlepage}


\AtBeginSection[]{

  \frame<beamer>{ 

    \frametitle{内容纲要}   

    \tableofcontents[currentsection] 

  }
}
\section{课程主要内容}

\begin{frame}
\frametitle{课程主要内容}
\begin{enumerate}
	\item { 线性方程组的直接解法}.
	\item { 线性最小二乘问题的数值算法}.
	\item { 非对称矩阵的特征值计算}.
	\item { 对称矩阵特征值计算与奇异值分解}.
	\item { 线性方程组迭代算法}.
	\item { 特征值问题的迭代算法 (部分特征值和特征向量)}. 
	\item { 稀疏矩阵计算}.  
\end{enumerate}
\end{frame}

\begin{frame}
\frametitle{主要参考资料}
\begin{enumerate}
	\item { G. H. Golub and C. F. van Loan, “Matrix Computations (4th),” 2013.对应中文版为:《矩阵计算》(第三版), 袁亚湘等译, 2001.}.
	\item { J. W. Demmel, “Applied Numerical Linear Algebra,” 1997.对应中文版为:《应用数值线性代数》, 王国荣译, 2007.}.
	\item { L. N. Trefethen and D. Bau, III, “Numerical Linear Algebra,” 1997.对应中文版为:《数值线性代数》, 陆金甫等译, 2006.}.
	\item { 徐树方, “矩阵计算的理论与方法,” 北京大学出版社, 1995.}.
	\item { 曹志浩, “数值线性代数,” 复旦大学出版社, 1996.}. 
\end{enumerate}

课程主页:

\qquad  http://math.ecnu.edu.cn/~jypan/Teaching/MatrixComp/

\end{frame}


\section{引言}
\begin{frame}
\frametitle{引言}
计算数学, 也称 数值分析 或 计算方法.


1947 年 Von Neumann 和 Goldstine 在《美国数学会通报》发表了题为“高阶矩阵的数值求逆”的著名论文, 开启了现代计算数学的研究


一般来说, 计算数学主要研究如何求出数学问题的近似解 (数值解), 包括算法的设计与分析 (收敛性, 稳定性, 复杂性等)
	
计算数学主要研究内容:
	
数值逼近, 数值微积分, 数值代数, 微分方程数值解, 最优化等
\end{frame}

\begin{frame}
\frametitle{为什么计算数学}
科学计算是 20 世纪重要科学技术进步之一, 已与理论研究和实验研究相并列成为科学研究的第三种方法. 现今科学计算已是体现国家科学技术核心竞争力的重要标志, 是国家科学技术创新发展的关键要素。

\qquad \qquad \qquad \qquad \qquad \qquad  —— 国家自然科学基金 · 重大项目指南, 2014

计算科学是 21 世纪确保国家核心竞争能力的战略技术之一。

\qquad \qquad \qquad \qquad \qquad \qquad    —— 计算科学: 确保美国竞争力,2005

科学计算的核心/数学基础: 计算数学.
\end{frame}

\begin{frame}
\frametitle{矩阵计算(数值线性代数)的重要性}
If any other mathematical topic is as fundamental to the mathematical
sciences as calculus and differential equations, it is numerical linear
algebra.

\qquad \qquad \qquad \qquad \qquad \qquad \qquad\qquad—— Trefethen and Bau, 1997.
\end{frame}

\begin{frame}
国家自然科学基金委员会关于计算数学的分类 (2018):

$\bullet$ 计算数学与科学工程计算 (A0117)

\qquad- 偏微分方程数值解 (A011701)

\qquad- 流体力学中的数值计算 (A011702)

\qquad- 偏微分方程数值解 (A011701)

\qquad- 流体力学中的数值计算 (A011702)

\qquad- 一般反问题的计算方法 (A011703)

\qquad- 常微分方程数值计算 (A011704)

\qquad- 数值代数 (A011705)

\qquad- 数值逼近与计算几何 (A011706)

\qquad- 谱方法及高精度数值方法 (A011707)

\qquad- 有限元和边界元方法 (A011708)

\qquad- 多重网格技术与区域分解 (A011709)

\qquad- 自适应方法 (A011720)

\qquad- 并行计算 (A011711)
\end{frame}

\begin{frame}
\frametitle{计算数学的主要任务}
$\bullet$算法设计: 构造求解各种数学问题的数值方法

$\bullet$算法分析: 收敛性、稳定性、复杂性、计算精度等

$\bullet$算法实现: 编程实现、软件开发

{\blue{好的数值方法一般需要满足以下几点:}}

$\bullet$有可靠的理论分析, 即收敛性、稳定性等有数学理论保证

$\bullet$有良好的计算复杂性 (时间和空间)

$\bullet$易于在计算机上实现

$\bullet$要有具体的数值试验来证明是行之有效的
\end{frame}

\begin{frame}
\frametitle{矩阵计算基本问题}
数值代数: 数值线性代数 (矩阵计算) 和数值非线性代数

数值线性代数 (矩阵计算) 主要研究以下问题:

\qquad$\bullet$线性方程组求解
$$Ax=b,A \in R^{n \times n}\text{非奇异}$$

\qquad$\bullet$(线性)最小二乘问题
$$
\min _{x \in \mathbb{R}^{n}}\|A x-b\|_{2} \quad, \quad A \in \mathbb{R}^{m \times n}, m \geq n
$$

\qquad$\bullet$矩阵特征值问题
$$
A x=\lambda x \quad, \quad A \in \mathbb{R}^{n \times n}, \lambda \in \mathbb{C}, x \in \mathbb{C}^{n}, x \neq 0
$$

\qquad$\bullet$矩阵奇异值问题
$$
A^{\top} A x=\sigma^{2} x \quad, \quad A \in \mathbb{R}^{m \times n}, \sigma \geq 0, x \in \mathbb{R}^{n}, x \neq 0
$$

\qquad$\bullet$其他问题:广义特征值问题, 二次特征值问题, 非线性特征值问题, 矩阵方程, 特征值反问题, 张量计算, . . . . . .

	{$\dagger$} 数值方法一般都是近似方法, 求出的解是带有误差的,因此误差分析非常重要.
	
\end{frame}

\begin{frame}
\frametitle{矩阵计算常用方法(技术或技巧)}
\qquad$\bullet$矩阵分解

\qquad$\bullet$矩阵分裂

\qquad$\bullet$扰动分析

{\red{$\dagger$}} 问题的特殊结构对算法的设计具有非常重要的影响.

{\red{$\dagger$}} 在编程实现时, 要充分利用现有的优秀程序库.
\end{frame}


\setcounter{equation}{0}
\frame{{二十世纪十大优秀算法}

{\scriptsize
\begin{enumerate}
  \item { Monte Carlo method (1946)}.
  \item { Simplex Method for Linear Programming (1947)}.
  \item { Krylov Subspace Iteration Methods (1950)}.
  \item { The Decompositional Approach to Matrix Computations (1951)}.
  \item {The Fortran Optimizing Compiler (1957)}.
  \item {QR Algorithm for Computing Eigenvalues (1959-61)}. 
  \item { Quicksort Algorithm for Sorting (1962)}. 
  \item {Fast Fourier Transform (1965)}.
  \item {Integer Relation Detection Algorithm (1977)}.
  \item {Fast Multipole Method (1987)}. 
\end{enumerate}
}
}

\end{document}
